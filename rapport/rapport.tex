\documentclass[10pt,a4paper]{article}
\usepackage[utf8]{inputenc}
\usepackage[francais]{babel}
\usepackage[T1]{fontenc}
\usepackage{hyperref}
\usepackage{eurosym}

\author{Anthony \textsc{Caccia} \and Romain \textsc{Fontaine} \and Nikita \textsc{Marchant} }
\date{}
\title{\textsc{INFO-F-309 : Administration Système} Projet : Rapport d'implémentation}

\setlength{\parindent}{1.5em}
\setlength{\parskip}{1em}
% \renewcommand{\baselinestretch}{1.5}

\begin{document}
\maketitle

\section{Serveur web (Machine Debian)}
\subsection{Services}
\begin{itemize}
    \item nginx \texttt{systemctl start/stop/status nginx.service}
    \item postgresql \texttt{systemctl start/stop/status nginx.service}
    \item redis \texttt{systemctl start/stop/status redis-server.service}
\end{itemize}

\subsection{Procédure d'installation}


\section{Serveur de compilation (Machine BSD)}
\subsection{Services}
\begin{itemize}
    \item clamAV :
        \begin{itemize}
            \item clamd \texttt{/usr/local/etc/rc.d/clamav-clamd start/stop}
            \item freshclam \texttt{/usr/local/etc/rc.d/clamav-freshclam ? start/stop}
        \end{itemize}
\end{itemize}

\subsection{Procédure d'installation}

En tant que root :
    pkg install bash zsh mosh git nginx python34 vim subversion screen cpdup py34-sqlite3 py27-psycopg2 clamav
    freebsd-update fetch
    freebsd-update install

Configuration du nfs :
    # ajouter `nfs_client_enable="YES"' dans /etc/rc.conf
    service nfsclient start # démarer le service sans devoir rebooter
    # ajouter la ligne suivante à /etc/fstab (l'ip est celle de la machine debian qui est le serveur nfs)
    # 192.81.220.4:/var/nfs        /mnt        nfs        rw        0        0
    mount -a # monte le nfs

Configuration de NTP :
    # ajouter `ntpd_enable="YES"' dans /etc/rc.conf
    service ntpd start

Configuration de Celery :
En root:
    adduser # commande interactive
    # répondre comme ceci :
    # Username: celery
    # Full name: celery
    # Shell (sh csh tcsh zsh rzsh git-shell bash rbash nologin) [sh]:  bash
    # Use password-based authentication? [yes]: no

    sudo python3.4 -m ensurepip # installe pip pour python3.4
    sudo python3.4 -m pip install virtualenv # install virtualenv

    # création des dossiers pour les pid et les logs de celery
    sudo mkdir /var/log/celery
    sudo chown celery:celery /var/log/celery/
    sudo mkdir /var/run/celery
    sudo chown celery:celery /var/run/celery/

En tant que celery :
    cd # retourner dans la home de celery

    git clone https://github.com/C4ptainCrunch/info-f-309.git
    cd info-f-309/webview

    virtualenv ve
    source ve/bin/activate
    pip install -r requirements.txt
    pip install psycopg2

    nano webview/local_settings.py # y placer le contenu de local_settings_bsd.py ci-joint

A nouveau en root :
    nano /etc/rc.d/celery #y placer le contenu de celery-rc.d ci-joint
    # ajouter `celery_enable="YES"' dans /etc/rc.conf
    # ajouter `celery_cmd="/home/celery/info-f-309/webview/ve/bin/celery"' dans /etc/rc.conf
    service celery start

Vous pouvez vérifier que celery tourne bien en allant regarder son pid dans \texttt{/var/run/celery/celery.pid}

Configuration de ClamAV:
En root :
    service clamav-freshclam onestart # démarage à la main de freshclam pour télécharger la base de signatures
    sleep 300 # attendre quelques minutes qu'elle soit téléchargée
    service clamav-clamd onestart # démarer le daemon à la main et vérifier qu'il accepte de lire la base de signatures
    # ajouter `clamav_freshclam_enable="YES"' et `clamav_clamd_enable="YES"' dans /etc/rc.conf


\end{document}

